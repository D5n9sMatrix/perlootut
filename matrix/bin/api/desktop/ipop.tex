% Options for packages loaded elsewhere
\PassOptionsToPackage{unicode}{hyperref}
\PassOptionsToPackage{hyphens}{url}
%
\documentclass[
]{article}
\usepackage{amsmath,amssymb}
\usepackage{lmodern}
\usepackage{ifxetex,ifluatex}
\ifnum 0\ifxetex 1\fi\ifluatex 1\fi=0 % if pdftex
  \usepackage[T1]{fontenc}
  \usepackage[utf8]{inputenc}
  \usepackage{textcomp} % provide euro and other symbols
\else % if luatex or xetex
  \usepackage{unicode-math}
  \defaultfontfeatures{Scale=MatchLowercase}
  \defaultfontfeatures[\rmfamily]{Ligatures=TeX,Scale=1}
\fi
% Use upquote if available, for straight quotes in verbatim environments
\IfFileExists{upquote.sty}{\usepackage{upquote}}{}
\IfFileExists{microtype.sty}{% use microtype if available
  \usepackage[]{microtype}
  \UseMicrotypeSet[protrusion]{basicmath} % disable protrusion for tt fonts
}{}
\makeatletter
\@ifundefined{KOMAClassName}{% if non-KOMA class
  \IfFileExists{parskip.sty}{%
    \usepackage{parskip}
  }{% else
    \setlength{\parindent}{0pt}
    \setlength{\parskip}{6pt plus 2pt minus 1pt}}
}{% if KOMA class
  \KOMAoptions{parskip=half}}
\makeatother
\usepackage{xcolor}
\IfFileExists{xurl.sty}{\usepackage{xurl}}{} % add URL line breaks if available
\IfFileExists{bookmark.sty}{\usepackage{bookmark}}{\usepackage{hyperref}}
\hypersetup{
  pdftitle={ipop.r},
  pdfauthor={denis},
  hidelinks,
  pdfcreator={LaTeX via pandoc}}
\urlstyle{same} % disable monospaced font for URLs
\usepackage[margin=1in]{geometry}
\usepackage{color}
\usepackage{fancyvrb}
\newcommand{\VerbBar}{|}
\newcommand{\VERB}{\Verb[commandchars=\\\{\}]}
\DefineVerbatimEnvironment{Highlighting}{Verbatim}{commandchars=\\\{\}}
% Add ',fontsize=\small' for more characters per line
\usepackage{framed}
\definecolor{shadecolor}{RGB}{248,248,248}
\newenvironment{Shaded}{\begin{snugshade}}{\end{snugshade}}
\newcommand{\AlertTok}[1]{\textcolor[rgb]{0.94,0.16,0.16}{#1}}
\newcommand{\AnnotationTok}[1]{\textcolor[rgb]{0.56,0.35,0.01}{\textbf{\textit{#1}}}}
\newcommand{\AttributeTok}[1]{\textcolor[rgb]{0.77,0.63,0.00}{#1}}
\newcommand{\BaseNTok}[1]{\textcolor[rgb]{0.00,0.00,0.81}{#1}}
\newcommand{\BuiltInTok}[1]{#1}
\newcommand{\CharTok}[1]{\textcolor[rgb]{0.31,0.60,0.02}{#1}}
\newcommand{\CommentTok}[1]{\textcolor[rgb]{0.56,0.35,0.01}{\textit{#1}}}
\newcommand{\CommentVarTok}[1]{\textcolor[rgb]{0.56,0.35,0.01}{\textbf{\textit{#1}}}}
\newcommand{\ConstantTok}[1]{\textcolor[rgb]{0.00,0.00,0.00}{#1}}
\newcommand{\ControlFlowTok}[1]{\textcolor[rgb]{0.13,0.29,0.53}{\textbf{#1}}}
\newcommand{\DataTypeTok}[1]{\textcolor[rgb]{0.13,0.29,0.53}{#1}}
\newcommand{\DecValTok}[1]{\textcolor[rgb]{0.00,0.00,0.81}{#1}}
\newcommand{\DocumentationTok}[1]{\textcolor[rgb]{0.56,0.35,0.01}{\textbf{\textit{#1}}}}
\newcommand{\ErrorTok}[1]{\textcolor[rgb]{0.64,0.00,0.00}{\textbf{#1}}}
\newcommand{\ExtensionTok}[1]{#1}
\newcommand{\FloatTok}[1]{\textcolor[rgb]{0.00,0.00,0.81}{#1}}
\newcommand{\FunctionTok}[1]{\textcolor[rgb]{0.00,0.00,0.00}{#1}}
\newcommand{\ImportTok}[1]{#1}
\newcommand{\InformationTok}[1]{\textcolor[rgb]{0.56,0.35,0.01}{\textbf{\textit{#1}}}}
\newcommand{\KeywordTok}[1]{\textcolor[rgb]{0.13,0.29,0.53}{\textbf{#1}}}
\newcommand{\NormalTok}[1]{#1}
\newcommand{\OperatorTok}[1]{\textcolor[rgb]{0.81,0.36,0.00}{\textbf{#1}}}
\newcommand{\OtherTok}[1]{\textcolor[rgb]{0.56,0.35,0.01}{#1}}
\newcommand{\PreprocessorTok}[1]{\textcolor[rgb]{0.56,0.35,0.01}{\textit{#1}}}
\newcommand{\RegionMarkerTok}[1]{#1}
\newcommand{\SpecialCharTok}[1]{\textcolor[rgb]{0.00,0.00,0.00}{#1}}
\newcommand{\SpecialStringTok}[1]{\textcolor[rgb]{0.31,0.60,0.02}{#1}}
\newcommand{\StringTok}[1]{\textcolor[rgb]{0.31,0.60,0.02}{#1}}
\newcommand{\VariableTok}[1]{\textcolor[rgb]{0.00,0.00,0.00}{#1}}
\newcommand{\VerbatimStringTok}[1]{\textcolor[rgb]{0.31,0.60,0.02}{#1}}
\newcommand{\WarningTok}[1]{\textcolor[rgb]{0.56,0.35,0.01}{\textbf{\textit{#1}}}}
\usepackage{graphicx}
\makeatletter
\def\maxwidth{\ifdim\Gin@nat@width>\linewidth\linewidth\else\Gin@nat@width\fi}
\def\maxheight{\ifdim\Gin@nat@height>\textheight\textheight\else\Gin@nat@height\fi}
\makeatother
% Scale images if necessary, so that they will not overflow the page
% margins by default, and it is still possible to overwrite the defaults
% using explicit options in \includegraphics[width, height, ...]{}
\setkeys{Gin}{width=\maxwidth,height=\maxheight,keepaspectratio}
% Set default figure placement to htbp
\makeatletter
\def\fps@figure{htbp}
\makeatother
\setlength{\emergencystretch}{3em} % prevent overfull lines
\providecommand{\tightlist}{%
  \setlength{\itemsep}{0pt}\setlength{\parskip}{0pt}}
\setcounter{secnumdepth}{-\maxdimen} % remove section numbering
\ifluatex
  \usepackage{selnolig}  % disable illegal ligatures
\fi

\title{ipop.r}
\author{denis}
\date{2021-07-12}

\begin{document}
\maketitle

\begin{Shaded}
\begin{Highlighting}[]
\CommentTok{\#!/usr/bin/r}

\CommentTok{\# However the data are represented, whether in an array or a network, the}
\CommentTok{\# analysis of the data is often facilitated by using “association” matrices. The}
\CommentTok{\# most familiar type of association matrix is perhaps a correlation matrix. We}
\CommentTok{\# will encounter and use other types of association matrices in Chapter 8.}

\CommentTok{\# In this chapter we discuss a wide range of basic topics related to vectors }
\CommentTok{\# of real}
\CommentTok{\# numbers. Some of the properties carry over to vectors over other fields, such}
\CommentTok{\# as complex numbers, but the reader should not assume this. Occasionally, for}
\CommentTok{\# emphasis, we will refer to “real” vectors or “real” vector spaces, but unless }
\CommentTok{\# it}
\CommentTok{\# is stated otherwise, we are assuming the vectors and vector spaces are real.}
\CommentTok{\# The topics and the properties of vectors and vector spaces that we emphasize}
\CommentTok{\# are motivated by applications in the data sciences.}
\NormalTok{real }\OtherTok{\textless{}{-}} \FunctionTok{vector}\NormalTok{(}\AttributeTok{mode =} \StringTok{"logical"}\NormalTok{, }\AttributeTok{length =}\NormalTok{ 0L)}
\NormalTok{real}
\end{Highlighting}
\end{Shaded}

\begin{verbatim}
## logical(0)
\end{verbatim}

\begin{Shaded}
\begin{Highlighting}[]
\CommentTok{\# Abstract}
\CommentTok{\# kernlab is an extensible package for kernel{-}based machine learning methods }
\CommentTok{\# in R. It takes}
\CommentTok{\# advantage of R’s new S4 object model and provides a framework for creating }
\CommentTok{\# and using kernel{-}}
\CommentTok{\# based algorithms. The package contains dot product primitives (kernels), }
\CommentTok{\# implementations}
\CommentTok{\# of support vector machines and the relevance vector machine, Gaussian }
\CommentTok{\# processes, a ranking}
\CommentTok{\# algorithm, kernel PCA, kernel CCA, kernel feature analysis, online kernel }
\CommentTok{\# methods and a}
\CommentTok{\# spectral clustering algorithm. Moreover it provides a general purpose }
\CommentTok{\# quadratic programming}
\CommentTok{\# solver, and an incomplete Chomsky decomposition method.}
\CommentTok{\# Keywords: kernel methods, support vector machines, quadratic programming, }
\CommentTok{\# ranking, clustering,}
\CommentTok{\# S4, R.}
\NormalTok{n }\OtherTok{=} \DecValTok{1}
\NormalTok{S4 }\OtherTok{\textless{}{-}} \FunctionTok{rnorm}\NormalTok{(n, }\AttributeTok{mean =} \DecValTok{0}\NormalTok{, }\AttributeTok{sd =} \DecValTok{1}\NormalTok{)}
\NormalTok{r }\OtherTok{=} \DecValTok{1}
\NormalTok{CCA }\OtherTok{\textless{}{-}} \FunctionTok{kernel}\NormalTok{(}\AttributeTok{coef =} \StringTok{"daniell"}\NormalTok{, }\AttributeTok{m =} \DecValTok{2}\NormalTok{, r, }\AttributeTok{name =} \StringTok{"unknown"}\NormalTok{)}
\NormalTok{S4}
\end{Highlighting}
\end{Shaded}

\begin{verbatim}
## [1] 1.015126
\end{verbatim}

\begin{Shaded}
\begin{Highlighting}[]
\NormalTok{CCA}
\end{Highlighting}
\end{Shaded}

\begin{verbatim}
## Daniell(2) 
## coef[-2] = 0.2
## coef[-1] = 0.2
## coef[ 0] = 0.2
## coef[ 1] = 0.2
## coef[ 2] = 0.2
\end{verbatim}

\begin{Shaded}
\begin{Highlighting}[]
\CommentTok{\# 1. Introduction}
\CommentTok{\# Machine learning is all about extracting structure from data, but it is often }
\CommentTok{\# difficult to solve prob{-}}
\CommentTok{\# lets like classification, regression and clustering in the space in which the }
\CommentTok{\# underlying observations}
\CommentTok{\# have been made.}
\CommentTok{\# Kernel{-}based learning methods use an implicit mapping of the input data into }
\CommentTok{\# a high dimensional}
\CommentTok{\# feature space defined by a kernel function, i.e., a function returning the }
\CommentTok{\# inner product hΦ(x), Φ(y)i}
\CommentTok{\# between the images of two data points x, y in the feature space. The learning }
\CommentTok{\# then takes place}
\CommentTok{\# in the feature space, provided the learning algorithm can be entirely }
\CommentTok{\# rewritten so that the data}
\CommentTok{\# points only appear inside dot products with other points. This is often }
\CommentTok{\# referred to as the “kernel}
\CommentTok{\# trick” (Schölkopf and Scold 2002). More precisely, if a projection }
\CommentTok{\# V : X → H is used, the dot}
\CommentTok{\# product hp(x), V(y)i can be represented by a kernel function k}
\CommentTok{\# k(x, y) = hp(x), V(y)i,}
\NormalTok{k }\OtherTok{\textless{}{-}} \ControlFlowTok{function}\NormalTok{(hp)\{}
\NormalTok{  i }\OtherTok{\textless{}{-}} \DecValTok{0}
\NormalTok{  hp }\OtherTok{\textless{}{-}} \FunctionTok{c}\NormalTok{(}\FunctionTok{list}\NormalTok{(i), }\DecValTok{0}\NormalTok{, i)}

  
  \FunctionTok{c}\NormalTok{(hp, i)}
\NormalTok{\}}
\FunctionTok{k}\NormalTok{(hp)}
\end{Highlighting}
\end{Shaded}

\begin{verbatim}
## [[1]]
## [1] 0
## 
## [[2]]
## [1] 0
## 
## [[3]]
## [1] 0
## 
## [[4]]
## [1] 0
\end{verbatim}

\begin{Shaded}
\begin{Highlighting}[]
\CommentTok{\# which is computationally simpler than explicitly projecting x and y into the }
\CommentTok{\# feature space H.}
\CommentTok{\# One interesting property of kernel{-}based systems is that, once a valid kernel }
\CommentTok{\# function has been}
\CommentTok{\# selected, one can practically work in spaces of any dimension without paying }
\CommentTok{\# any computational}
\CommentTok{\# cost, since feature mapping is never effectively performed. In fact, one does }
\CommentTok{\# not even need to know}
\CommentTok{\# which features are being used.}
\CommentTok{\# Another advantage is the that one can design and use a kernel for a particular }
\CommentTok{\# problem that could be}
\CommentTok{\# applied directly to the data without the need for a feature extraction }
\CommentTok{\# process. This is particularly}
\CommentTok{\# important in problems where a lot of structure of the data is lost by the }
\CommentTok{\# feature extraction process}
\CommentTok{\# (e.g., text processing). The inherent popularity of kernel{-}based learning }
\CommentTok{\# methods allows one to}
\CommentTok{\# use any valid kernel on a kernel{-}based algorithm.}
\NormalTok{feature }\OtherTok{\textless{}{-}} \FunctionTok{c}\NormalTok{(}\DecValTok{10}\NormalTok{, }\AttributeTok{type =} \FunctionTok{c}\NormalTok{(}\StringTok{"O"}\NormalTok{, }\StringTok{"I"}\NormalTok{, }\StringTok{"F"}\NormalTok{, }\StringTok{"M"}\NormalTok{, }\StringTok{"2"}\NormalTok{))}
\NormalTok{feature}
\end{Highlighting}
\end{Shaded}

\begin{verbatim}
##       type1 type2 type3 type4 type5 
##  "10"   "O"   "I"   "F"   "M"   "2"
\end{verbatim}

\begin{Shaded}
\begin{Highlighting}[]
\CommentTok{\# 1.1. Software review}
\CommentTok{\# The most prominent kernel based learning algorithm is without doubt the }
\CommentTok{\# support vector machine2}
\CommentTok{\# kernlab – An S4 Package for Kernel Methods in R}
\CommentTok{\# (SVM), so the existence of many support vector machine packages comes as }
\CommentTok{\# little surprise. Most}
\CommentTok{\# of the existing SVM software is written in C or C++, e.g. the award winning }
\CommentTok{\# libsvm 1 (Chang and}
                                                                                      
\CommentTok{\# Lin 2001), Sunlight 2 (Joachims 1999), SVMTorch 3 , Royal Holloway Support }
\CommentTok{\# Vector Machines 4 ,}
\CommentTok{\# myS 5 , and M{-}SVM 6 with many packages providing interfaces to }
\CommentTok{\# MATLAB (such as libsvm),}
\CommentTok{\# and even some native MATLAB toolboxes 7 8 9 .}
\CommentTok{\# Putting SVM specific software aside and considering the abundance of other }
\CommentTok{\# kernel{-}based ago{-}}
\CommentTok{\#  rhythms published nowadays, there is little software available implementing }
\CommentTok{\# a wider range of kernel}
\CommentTok{\# methods with some exceptions like the Spider 10 software which provides }
\CommentTok{\# a MATLAB interface to}
\CommentTok{\# various C/C++ SVM libraries and MATLAB implementations of various }
\CommentTok{\# kernel{-}based algorithms,}
\CommentTok{\# Torch 11 which also includes more traditional machine learning algorithms, }
\CommentTok{\# and the occasional}
\CommentTok{\# MATLAB or C program found on a personal web page where an author includes }
\CommentTok{\# code from a}
\CommentTok{\# published paper.}
\NormalTok{kernlab}\SpecialCharTok{::}\NormalTok{.\_\_C\_\_ipop}
\end{Highlighting}
\end{Shaded}

\begin{verbatim}
## Class "ipop" [package "kernlab"]
## 
## Slots:
##                                     
## Name:     primal      dual       how
## Class:    vector   numeric character
\end{verbatim}

\end{document}
